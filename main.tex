%=================================================================
%				preamble
%=================================================================
\documentclass[11pt,letterpaper]{book}

\newcommand{\theauthor}{Thomas Graf}
\newcommand{\lastname}{Graf}
\newcommand{\university}{Stony Brook University}	
\newcommand{\emailaddress}{lin637@thomasgraf.net}
\newcommand{\coursenumber}{LIN 637}
\newcommand{\coursename}{Computational Linguistics as Language Science}
\newcommand{\semester}{Spring 2020}
\newcommand{\thetitle}{\texorpdfstring{%
            {\Huge \coursename}\\
            \semester}%
            {\coursename\ [\coursenumber, \semester]}}
\newcommand{\thekeywords}{graduate level, lecture, computational linguistics, phonology, morphology, syntax}
\newcommand{\thedate}{}

\usepackage{mypackages}
\usepackage{mycommands}

\addbibresource{bib/graf.bib}
\addbibresource{bib/universal.bib}

%=================================================================
%			title format
%=================================================================
\author{\theauthor}
\title{\thetitle}
\date{\thedate}

%=================================================================
%			content
%=================================================================
\pagestyle{empty}
% \includeonly{./tex/preface,./tex/formalization}

\begin{document}
\maketitle
\raggedbottom
\pagenumbering{Roman}
\pagestyle{fancy}

\chapter*{About this book}


\section*{What?}

This textbook is a graduate-level introduction to proof-based (rather than modeling-based) computational linguistics for linguists.
The focus is on how a computationally informed perspective can illuminate aspects of language that theoretical linguists and cognitive scientists care about.
A concerted effort has been made to keep the material as approachable as possible without sacrificing exactness.
To get the most out of the book, readers have to be willing to engage with mathematical notation, but each unit is designed in a modular fashion so that the less mathematically inclined can skip the parts they find too tedious.


\section*{Why?}

Like most textbooks, this one was written because of a picky instructor who wasn't quite happy with the existing options.
It grew out of my lecture notes for \emph{Computational Linguistics 2} at Stony Brook University's Department of Linguistics.
Within linguistics, computational linguistics tends to be geared towards model-building and simulations: MaxEnt learners, corpus-based techniques, and so on.
These topics are covered in a different course at Stony Brook, though, with Computational Linguistics 2 exploring the proof-based side of the field: formal language theory, subregular complexity, logic, string and tree automata\slash transducers, learnability, parsing theory, algebra, plus some algorithms and data structures.
There is a number of excellent textbooks that cover a subset of these topics, but they all fell short in some respect:

\begin{itemize}
    \item Marcus Kracht's \emph{The Mathematics of Language} is a treasure trove and still one of the most rewarding textbooks ever written, but it is too hard and theoretical for the average linguistics student.
    \item \emph{Speech and Language Processing} by Jurafsky and Martin is an excellent reference book, but it is clearly aimed at computer science students.
          The focus is on engineering techniques rather than investigating linguistic questions from a computational perspective.
    \item András Kornai's \emph{Mathematical Linguistics} presents a more appropriate compromise between linguistic inquiry and applications, but is still too far removed from linguistics and cognition for my taste.
    \item \emph{Mathematical Methods in Linguistics} by Partee, ter Meulen \& Wall is a classic, but as the title implies it mostly focuses on mathematics, in particular those areas that are needed for semantics (set theory, algebra, and predicate logic).
          Its final chapter on formal language theory is a commendable inclusion, but does not make a strong case for why linguists should care about this perspective.
    \item Ed Keenan and Larry Moss have produced an impressive and very comprehensive introduction to mathematical linguistics with \emph{Mathematical Structures in Language} (and I'm not just saying that because of the many fond memories I have as a teaching assistant for the course that their book sprang from).
          But overall it is a mathematical methods book, not an introduction to proof-based computational linguistics as a means to study language. 
    \item Some subtopics have dedicated textbooks, e.g. Laura Kallmeyer's \emph{Parsing Beyond Context-Free Grammars}, \emph{Categorial Grammar} by Glyn Morrill, \emph{The Logic of Categorial Grammars} by Richard Moot and Christian Retoré, or my colleague Jeff Heinz's book \emph{Grammatical Inference for Computational Linguistics}, co-authored with Colin de La Higuera and Menno van Zaanen.
          I have happily used them in special topics courses, but they proved difficult to sample from for a broad introduction course like Computational Linguistics 2. 
\end{itemize}

I am sure I have forgotten a few others, all of them deserving of a shout-out here.
But the bottom-line is that none fit my vision of proof-based computational linguistics as a way of studying language.
So here we are.


\section*{How?}

Computational linguistics is a very interdisciplinary field, and this is also reflected in the potential readership for a book like this: full-fledged theoretical linguists, computer scientists with no background in linguistics at all, or cognitive scientists that care about language only to the extent that it reveals deeper principles of human cognition.
It is impossible to serve all of them equally well, so I decided to keep the focus on what would benefit and interest linguists the most.
But I did make minor additions to include other groups --- that is why the linguistically trained reader will sometimes encounter brief sections on very basic material they obviously know about already.
I hope they won't feel patronized.

Even the core audience of linguistics grad student is far from homogeneous, though.
Depending on which classes they have taken before, their mathematics and programming background may range from non-existant to far above average.
For this reason each unit of the textbook adopts the non-linear design indicated in Fig.~\ref{fig:preface_generalstructure}.
This should allow readers (and instructors) to tailor the book to their specific background.

\begin{figure}
    \centering
    \tikzload{overview_unit}
\caption{General structure of each unit; dashed paths are optional}
\label{fig:preface_generalstructure}
\end{figure}

\section*{So what do I read?}

Every reader should be able to complete a unit's main path from the introduction to the summary.
While some math is unavoidable even in those general sections, the more demanding parts are put in a separate section.
All the math that is required to follow the main discussion is explained in separate background boxes:

\begin{techinfo}
These boxes occur throughout the text and introduce basic concepts of mathematics and computer science.
\end{techinfo}

As can be seen in Fig.~\ref{fig:preface_generalstructure}, the book also includes optional discussions of programming issues.
No, this is not a programming book.
But it is often instructive to analyze how exactly formal ideas can be translated into concrete code.
In particular, issues that linguists care about may appear under an entirely new light when one considers the concrete implementation challenges they pose.
I decided to go with Python instead of pseudo code for these sections, simply because most students nowadays are more likely to have previous experience with Python than with reading pseudo code.
To further increase the approachability of the code sections, I have adopted a beginner-friendly coding style.
I err very much on the verbose side with line comments and docstrings (which document the purpose and overall design of specific pieces of code).
I also make a deliberate effort to avoid advanced techniques such as decorators and generators.
Error handling, which is very important for production-level software, is completely omitted.
Similarly, efficiency is considered only to the extent that it illustrates a key pedagogical point.
Quite simply, the code sections are of little interest on their own and are designed only to support the theoretical concepts covered in each unit.

Depending on which parts of a unit the reader decides to skip, the book can range from a light-weight introduction that conveys the key intuitions to a very formal \emph{tour de force}.


\section*{No, which units should I read?}

Why, all of them!
There are several shorter paths through the book, but you'll have to figure out on your own which one suits your needs.
All I can do is provide a Hasse diagram to track the dependencies between the units.

fixme: include diagram for whole book


\section*{Are there exercises? Are there solutions?}

Exercises are included without solution so that they can be reused by instructors in their course.
Each exercise is assigned one of four difficulty levels, indicated by asterisks:

\begin{homework}
    The lowest difficulty.
    Every student should be able to solve this by simply applying a key technique of the unit in a mechanical fashion.
\end{homework}

\begin{homework}[*]
    Requires some effort and creativity.
    The student cannot simply follow a recipe from the unit but has to build on their understanding of the material to come up with a solution of their own.
\end{homework}

\begin{homework}[**]
    Very difficult for students, and instructors might have to sit down for a minute or two to figure out the general strategy.
    Writing up the full solution might take quite a while, and\slash or the solution involves a lot of lateral, out-of-the-box thinking.
\end{homework}

\begin{homework}[***]
    Open research problem.
    If somebody figures out the answer, they should write a paper about it.
    Or send me the solution and I'll handle the tedious paper publishing part for you.
\end{homework}


\section*{What's next?}

You tell me.
Go find yourself a problem and solve it.
If you need inspiration, pick some of the papers that are mentioned in the literature discussion sections --- you should be able to follow along fairly well after having read this book. 
Maybe you're doing fieldwork on an understudied language that has a surprisingly complex phenomenon.
You might feel the urge to explore the formal properties of a specific linguistic formalism that's dear to your heart but wasn't covered here.
Perhaps you'll go and prove an open conjecture, or you find a crucial mistake in a published proof.
Maybe, just maybe, you'll use an idea sketched in this book and expand it into a piece of technology that will make you millions (ideally in a currency where that is an above-average amount of money).
You're the next generation.
Surprise me.
Surprise yourself.

\clearpage

\shorttoc{Contents}{0}
\clearpage

\tableofcontents
\clearpage

% \setcounter{chapter}{-1}
\chapter{Syllabus}
\label{cha:syllabus}
\setcounter{page}{1}

\fcolorbox{gray!25}{gray!25}{%
    \centering
    \begin{tabular}{ll}
        \textbf{Course:} Computational Linguistics 2&
        \textbf{Name:} Thomas Graf\\
        \textbf{Course\#:} LIN 637 &
        \textbf{Email:} \href{mailto:lin637@thomasgraf.net}{lin637@thomasgraf.net}\\
        \textbf{Time:} R 5:30--7:30 &
        \textbf{Office hours:} M 11:15--12:00, F 11:15-12:00\\
        \textbf{Location:} CompLab SBS N250 &
        \textbf{Office:} SBS N249\\
        \textbf{Course Website:} \href{http://lin637.thomasgraf.net} {lin637.thomasgraf.net} &
        \textbf{Personal Website:} \href{http://thomasgraf.net}{thomasgraf.net}
    \end{tabular}
}

\section{Course Outline}

\subsection{Bulletin Description}
An introduction to the theoretical foundation of computational linguistics.
The course emphasizes the importance of algorithms, algebra, logic, and formal language theory in the development of new tools and software applications.
Empirical phenomena in phonology and syntax are sampled from a variety of languages to motivate and illustrate the use of concepts such as strictly local string languages, tree transducers, and semirings.
Students will develop familiarity with the literature and tools of the field.

\subsection{Full Description}

This course serves a specific purpose in our program (see Fig.~\vref{fig:Syllabus_Program}):
it acts as the bridge from introductory courses in linguistics (Syntax 1, Phonology 1, Phonetics) and computational methods (Statistics, Mathematical Methods in Linguistics, Computational Linguistics 1) to advanced courses and seminars in computational\slash mathematical linguistics.
In contrast to the NLP courses offered by the department of computer science, this course focuses on studying the properties of natural language from a computationally informed perspective.
The question is not how computers can solve language-related tasks, but how language can be conceptualized as a computational problem.
This emphasis is also reflected in the selection of topics for this course.

\begin{itemize}
    \item \textbf{What this course is not about}
        \begin{itemize}
            \item Computer-assisted research methods in linguistics
            \item Software development for natural language tasks
        \end{itemize}
    \item \textbf{What is not covered but benefits from what is covered}
        \begin{itemize}
            \item Speech recognition
            \item OCR
            \item Text generation
            \item Parsing
            \item Semantic analysis
            \item Machine translation
        \end{itemize}
    \item \textbf{List of topics}
        \begin{itemize}
            \item \emph{Phonology and Morphology}
                \begin{itemize}
                    \item The role of formalization
                    \item String languages
                    \item Subregular hierarchy
                    \item Regular languages
                    \item Generative capacity of phonology
                    \item String transductions
                    \item 2-level morphology
                    \item Equivalence of SPE and OT
                \end{itemize}
            \item \emph{Syntax}
                \begin{itemize}
                    \item Tree languages
                    \item Syntax is more complex than phonology
                    \item Mildly context-sensitive formalisms (TAG, MGs)
                    \item Tree transductions
                    \item Regular representations of MCS formalisms
                    \item Reinterpreting the T-model
                \end{itemize}
        \end{itemize}
\end{itemize}

A rough outline of the course progression is given in Tab.~\vref{tab:Syllabus_CourseOutline}.
Many of the topics we cover draw from very specialized areas of formal language theory that even most mathematicians and computer scientists do not know about, e.g.\ the correspondence between finite-state machines and monadic second-order logic, or the logical characterization of tree transductions.
So this course might be of value to you even if you do not particularly care about natural language.
Make no mistake, though, we'll talk a lot about language and linguistics --- this is not a math class.

\begin{table}
    \centering
    \begin{tabular}{rp{5.5cm}p{5.5cm}}
        \toprule
        \emph{Wk} & \emph{Formal} & \emph{Linguistics} \\
        \toprule
        1         & What is computation? & Marr's Three Levels\\
        2         & Formalizing phonology & Why formalize?\\
        3         & Strictly local languages & Local dependencies\\
        4         & Subregular hierarchy & How powerful is phonology?\\
        5         & Regular languages & Abstractness\\
        6         & String transductions & SPE-OT equivalence\\
        7         & Two-level morphology & Null morphemes\\
        \midrule
        8         & (Spring Break) & \\
        \midrule
        9         &  Weak Generative Capacity & $\text{Phonology} < \text{Syntax}$\\
        10        &  Tree languages & Headedness, feature percolation\\
        11        &  Local tree languages & GPSG\\
        12        &  Recognizable tree languages & GB\\
        13        &  TAG and MGs & Minimalist syntax\\
        14        &  Tree transductions & Reinterpreting the T-model\\
        15        &  Unification via first-order logic & Strictly Derivational Minimalism\\
        \bottomrule
    \end{tabular}
\caption{Tentative course outline}
\label{tab:Syllabus_CourseOutline}
\end{table}

\begin{figure}
    \rotatebox{0}{
        \footnotesize
        \input{./img/tikz/ProgramOverview.tex}
    }
\caption{Computational Linguistics 2 in the curriculum (dashed lines indicate recommendations rather than prerequisites)}
\label{fig:Syllabus_Program}
\end{figure}    

\subsection{Prerequisites}

The only official prerequisite is Computational Linguistics 1 (Lin 537) or comparable programming skills in Python.
Python will be used to illustrate formal concepts, and some of the homeworks will require you to implement an algorithm or procedure in Python.
Prior experience with git and markdown is useful for the homeworks but not required.

It is also helpful to have some basic familiarity with linguistics (phonemes, phrase structure rules, syntactic trees) and mathematics (sets, functions, relations, and first-order logic as covered in Semantics 1, for instance).
You can take an online survey to identify weaknesses, and several introductory readings on these topics are available on the course website.

\medskip
\noindent
\hspace{-.75em}
\begin{tabular}{ll}
    \textbf{Survey URL:} & 
    \href{https://testmoz.com/432409}{https://testmoz.com/432409}
\end{tabular}

\section{Teaching Goals}
\begin{itemize}
    \item \textbf{Practical Skills}
        \begin{itemize}
            \item conceptualize a problem in mathematical terms
            \item optimize your programs through the use of adequate algorithms and data structures (dynamic programming techniques, hash tables, etc.)
            \item a more abstract and theoretically informed perspective on current tools and techniques in NLP
            \item an understanding for how linguistic insights can be invoked to simplify NLP tasks
        \end{itemize}
    \item \textbf{Research Skills}
        \begin{itemize}
            \item assess linguistic phenomena from a computational perspective
            \item evaluate linguists' claims about computational efficiency
            \item basic overview of current research in theoretical computational linguistics
            \item use computational concepts to identify new empirical generalizations
            \item bring linguistic data to bear on computational claims
            \item mathematically informed understanding of linguistic theories
        \end{itemize}
\end{itemize}


\section{Grading}
\begin{itemize}
    \item \textbf{Homework}
        \begin{itemize}
            \item exercises, programming assignments, or critical evaluations of assigned readings
            \item Homework submission and grading is done via github.
            \item No late hand-ins!
            \item Collaboration on homework problems is encouraged as long as you write up the solutions by yourself, using your own words, examples, notation, and code.
        \end{itemize}
        %
    \item \textbf{Readings}
        \begin{itemize}
            \item at most two readings per week
            \item It is presupposed in the lectures that you have done the required readings.
            \item Reading comprehension may be tested as parts of the homeworks.
        \end{itemize}
        %
    \item \textbf{Lecture Note Feedback}
        \begin{itemize}
            \item I plan to publish the lecture notes as an open-access textbook with \emph{Language Science Press}.
            \item This is the last time I teach the course before the submission deadline, so I want feedback.
            \item Every week, you should file issues on Github for the relevant units, where you spot typos, suggest exercises, pictures, examples, etc.
            \item This is a collaborative enterprise: comment on other student's suggestions if you (dis)approve, expand their ideas, and so on.
        \end{itemize}
        %
    \item \textbf{Workload per Credits}
        \begin{itemize}
            \item \emph{0 credits}: none, but I highly recommend that you at least read the assigned papers as they will be important for following the lectures
            \item \emph{1 credit}: readings
            \item \emph{2 credits}: readings, feedback
            \item \emph{3 credits}: homework, readings, feedback
        \end{itemize}
\end{itemize}


\section{Online Component}

This class uses some online tools to facilitate homework collaboration and submission, student discussions, and dynamic lecture evaluation.

\begin{itemize}
    \item \textbf{Homework submission}\\
        \emph{How it works:}
        Homeworks will distributed via a github repository. 
        You can fork this repo and upload your own code, or checkout other students' forks to see how they dealt with the problem.
        In order to submit a homework you upload your solution to your fork and issue a pull request.
        After the due date, I'll upload my solution to the repository.

        \emph{Why we do it:}
        This setup mimics the modern workflow in collaborative development projects.
        Git is one of the best-known version control systems, and github is the biggest online service for hosting git repositories.
        Familiarity with version control systems is an essential job requirement for computational linguists, and it is also very helpful for academic work.
        See this discussion on Stackflow for some ideas how git can be used in conjunction with Latex:
        \href{http://stackoverflow.com/questions/6188780/git-latex-workflow}{http://stackoverflow.com/questions/6188780/git-latex-workflow}

        \emph{What you'll need:}
        A github account (the free tier is enough) and a way of uploading your code to a github repository.
        Linux users can install git via the command line, whereas Windows and Mac users should download and install the github app, which comes with a nice GUI.

    \item \textbf{Homework Feedback and Discussion}\\
        \emph{How it works:}
        Every github repository comes with an issue (= ticket) tracker.
        If you have a question or wish to discuss a topic, you can open a ticket on the main repository (note: tickets support markdown).
        I may also use tickets to leave comments on your homeworks.

        \emph{Why we do it:}
        Once again this is an essential part of modern software development.
        And it is also a lot more convenient than anything Blackboard has to offer.

        \emph{What you'll need:}
        Not much beyond the ability to navigate the github repos.

    \item \textbf{Class announcements}\\
        \emph{How it works:}
        Normal announcements (readings, due dates) are put on the course website, time-critical ones (i.e.\ class cancellations) are emailed out via Blackboard.
        
        \emph{What you'll need:}
        If you're not officially enrolled in the course for at least 0 credits, send me a message so I can add you to Blackboard.

    \item \textbf{Lecture Notes}\\
        \emph{How it works:}
        The lecture notes are made available online Monday and Wednesday before 3pm.
        You can look at them on your laptop\slash tablet or make a hardcopy before class.
        I will not bring handouts to class unless I could not upload them on time the day before.

        \emph{Why we do it:}
        Mostly because I just don't like paper.
        Also keep in mind that you can fork the lecture notes repository and include your notes directly in the Latex source files.
        Then you can compile a version of the lecture notes with your own notes already included.
\end{itemize}


\section{Policies}

\subsection{Contacting me}
\begin{itemize}
    \item Emails should be sent to \href{mailto://lin637@thomasgraf.net}{lin637@thomasgraf.net} to make sure they go to my high priority inbox.
        Disregarding this policy means late replies and is a sure-fire way to get on my bad side.
    \item Reply time < 24h in simple cases, possibly more if meddling with bureaucracy is involved.
    \item If you want to come to my office hours and anticipate a longer meeting, please email me so that we can set apart enough time and avoid collisions with other students.
\end{itemize}

\subsection{Disability Support Services}

If you have a physical, psychological, medical or learning disability that may impact your course work, please contact Student Accessibility Support Center, ECC (Educational Communications Center) Building, Room 128, (631)632-6748.
They will determine with you what accommodations, if any, are necessary and appropriate.
All information and documentation is confidential.

Students who require assistance during emergency evacuation are encouraged to discuss their needs with their professors and Student Accessibility Support Center.
For procedures and information go to the following website:
\url{http://www.stonybrook.edu/ehs/fire/disabilities}

\subsection{Academic Integrity}

Each student must pursue his or her academic goals honestly and be personally accountable for all submitted work.
Representing another person's work as your own is always wrong. Faculty is required to report any suspected instances of academic dishonesty to the Academic Judiciary.
Faculty in the Health Sciences Center (School of Health Technology \& Management, Nursing, Social Welfare, Dental Medicine) and School of Medicine are required to follow their school-specific procedures.
For more comprehensive information on academic integrity, including categories of academic dishonesty please refer to the academic judiciary website at
\url{http://www.stonybrook.edu/commcms/academic_integrity/index.html}.

\subsection{Critical Incident Management}

Stony Brook University expects students to respect the rights, privileges, and property of other people.
Faculty are required to report to the Office of University Community Standards any disruptive behavior that interrupts their ability to teach, compromises the safety of the learning environment, or inhibits students' ability to learn.
Faculty in the HSC Schools and the School of Medicine are required to follow their school-specific procedures.
Further information about most academic matters can be found in the Undergraduate Bulletin, the Undergraduate Class Schedule, and the Faculty-Employee Handbook. 


\begin{literature}
add more info
\end{literature}

\pagenumbering{arabic}


\setcounter{chapter}{-1}
\chapter{Computation and formalization}
\label{cha:Formal}

Without a doubt the most unfortunate fact about the field of computational linguistics is that its name is \emph{computational linguistics}.
The term inherits a dichotomy that is hard to tease apart for the uninitiated: the distinction between computers and computing.

Computational linguistics can certainly be about solving language-related tasks with computers, but it can be so much more.
Once one understands how broad the notion of computation is, it becomes clear that the reach of computational linguistics can go far beyond computers.
In doing so, it provides a deeper understanding of what language looks like from a computational perspective.
This goal sounds rather vague, for sure, but it is not without merit.
% Much like a physicist seeks a deeper understanding of the laws of the universe and thus opens up the road towards new technologies, the computational study of language can help with the more grounded concerns of making computers language-savvy.

The book you are holding in your hands (and, presumably, reading) is all about this broader notion of computational linguistics.
The rest of this chapter tries to sharpen the profile of what we may call ``computational computational linguistics'', or simply \emph{computational$^2$ linguistics}.
We will also discuss why this approach is worth pursuing, lest you put away the book before even finishing its first chapter.
The specific merits of computational$^2$ linguistics depends a lot on whether you are a natural language engineer, a theoretical linguist, or a cognitive scientist, but each group stands to gain something.


\section{Computers vs computation}

Here is a statement that the average layperson will find puzzling:
While computers are the most common tool for carrying out computations nowadays, they are not what computation is about.
Computation, in its barest form, is the principled manipulation of information, of transforming some input into some output in a precise, step-wise fashion.
When a computer verifies $1 + 1 = 2$, this act of computation is instantiated via a series of electrical impulses that affect some of the millions of transistors that make up its hardware.
But the computation is not tied to that specific electrical process, it can take many physical instantiations in this world.

The movie buffs among you will remember the 90s masterpiece \emph{MacGyver: Lost Treasure of Atlantis}.
In this spiritual successor to the \emph{Indiana Jones} movies, the adventurer MacGyver discovers the secret of Atlantis: a giant, steam-powered computer that operates without electricity.
Of course one should not lend too much credence to a movie where the protagonist cruises through a military camp with a rocket car that he built in 20 minutes.
But the idea of a steam-powered computer is not nearly as outlandish.
Any device that can assume multiple different states and switch between those states in a controlled fashion is capable of computation.

The idea that computation is about transitioning from one state to another in a principled fashion is the central insight behind the \emph{Turing machine}.
% The Turing machine is named after Alan Turing (1912--1954), arguably one of the most brilliant and influential thinkers of the 20th century.
% It is impossible to do Turing justice in a few sentences, so reading up on his many accomplishments is left as an exercise to the reader.
A Turing machine consists of a tape, a read/write head, and what's called a state register.
The read/write head can move to any position on the tape, read the symbol that is there, and possibly overwrite it with a new symbol.
The state register is like a knob that can be in one of finitely many positions, e.g.\ 7 out of 10 on a volume dial.
The purpose of the state register is to control the behavior of the Turing machine.
Based on the symbol that is currently under the read/write head and the state of the register, the Turing machine performs a \emph{write action} (overwrite or do nothing), a \emph{move action} (move left, move right, stay in place), and a \emph{state register change} (keep state, switch to different state).
So a basic instruction of a Turing machine may read like ``if the symbol under the head is 1 and the state is A, overwrite 1 with 0, move left, and switch to state B''.
A finite collection of such instructions is a program that can be run on a Turing machine to carry out specific computations.

\begin{examplebox}[Copying with a Turing machine]
    The table below describes a small program for a Turing machine.
    \begin{center}
        \begin{tabular}{cc|ccc}
            \emph{state} & \emph{tape symbol} & \emph{write action} & \emph{move action} & \emph{new state}\\
            \hline
            A            & 0                  & none                & none               & F\\
            A            & 1                  & write(0)            & $\Leftarrow$       & B\\
            B            & 0                  & none                & $\Leftarrow$       & C\\
            B            & 1                  & none                & $\Leftarrow$       & B\\
            C            & 0                  & write(1)            & $\Rightarrow$      & D\\
            C            & 1                  & none                & $\Leftarrow$       & C\\
            D            & 0                  & none                & $\Rightarrow$      & E\\
            D            & 1                  & none                & $\Rightarrow$      & D\\
            E            & 0                  & write(1)            & $\Leftarrow$       & A\\
            E            & 1                  & none                & $\Rightarrow$      & E\\
        \end{tabular}
    \end{center}
    % 
    This looks fairly cryptic, so let's tease apart what's going on here.

    Each individual instruction is easy to decypher.
    The machine has 6 different states: A, B, C, D, E, and F.
    There are only two kinds of symbols are used on the tape, 0 and 1.
    A command like write(1) means that the machine writes a 1 onto the tape, whereas the arrows $\Leftarrow$ and $\Rightarrow$ specify that the machine moves one symbol to the left or right after the write action is finished.
    So line 5, for example, tells us that if the machine is in state C and has a 0 under its read/write head, it writes a 1, moves to the next symbol to the right, and switches into state D.

    That is all nice and dandy, but what is it the program does?
    This feels like a badly written instruction manual where each individual sentence makes sense but you can't figure out how they fit together (very much unlike this book, I hope).
    Let us look at a concrete example.

    Suppose that the machine starts in the following configuration: 
    The tape consists mostly of 0s, except for two adjacent 1s, and the read-write head is positioned on the right 1, with the state register in state A.
    This is visualized below.
    %
    \begin{center}
        \begin{tikzpicture}
            \matrix (m) at (0,0) [matrix of nodes] {
                0 & 0 & 0 & 0 & 1 & 1 & 0 & 0 \\
                  &   &   &   &   & A\\
            };

            \begin{pgfonlayer}{background}
                \draw[draw=blue!75,fill=blue!25,thick] (m-2-6.south west) rectangle (m-1-6.north east);
            \end{pgfonlayer}
        \end{tikzpicture}
    \end{center}
    %
    This configuration is matched by the second line of the instruction table.
    Hence the machine overwrites the current symbol with a 0, moves to the left and switches into state B\@.
    Here is the resulting configuration.
    %
    \begin{center}
        \begin{tikzpicture}
            \matrix (m) at (0,0) [matrix of nodes] {
                0 & 0 & 0 & 0 & 1 & 0 & 0 & 0 \\
                  &   &   &   & B \\
            };

            \begin{pgfonlayer}{background}
                \draw[draw=blue!75,fill=blue!25,thick] (m-2-5.south west) rectangle (m-1-5.north east);
            \end{pgfonlayer}
        \end{tikzpicture}
    \end{center}
    %
    Since the read/write head is now over a 1 while the machine is in state B, the instruction on line 4 is triggered: the machine keeps the current symbol as is, moves to the left, and keeps the register in state B\@.
    %
    \begin{center}
        \begin{tikzpicture}
            \matrix (m) at (0,0) [matrix of nodes] {
                0 & 0 & 0 & 0 & 1 & 0 & 0 & 0 \\
                  &   &   & B \\
            };

            \begin{pgfonlayer}{background}
                \draw[draw=blue!75,fill=blue!25,thick] (m-2-4.south west) rectangle (m-1-4.north east);
            \end{pgfonlayer}
        \end{tikzpicture}
    \end{center}
    %
    This in turn triggers the third instruction, so that the machine does not perform any write action, moves one symbol to the left, and switches the register to state C\@. 
    %
    \begin{center}
        \begin{tikzpicture}
            \matrix (m) at (0,0) [matrix of nodes] {
                0 & 0 & 0 & 0 & 1 & 0 & 0 & 0 \\
                  &   & C & \\
            };

            \begin{pgfonlayer}{background}
                \draw[draw=blue!75,fill=blue!25,thick] (m-2-3.south west) rectangle (m-1-3.north east);
            \end{pgfonlayer}
        \end{tikzpicture}
    \end{center}
    %
    The rest of the computation then proceeds as depicted below:
    %
    \begin{center}
        \begin{tikzpicture}
            \matrix (m) at (0,0) [matrix of nodes] {
                0 & 0 & 1 & 0 & 1 & 0 & 0 & 0 \\
                  &   & &   D\\
            };

            \begin{pgfonlayer}{background}
                \draw[draw=blue!75,fill=blue!25,thick] (m-2-4.south west) rectangle (m-1-4.north east);
            \end{pgfonlayer}
        \end{tikzpicture}

        \begin{tikzpicture}
            \matrix (m) at (0,0) [matrix of nodes] {
                0 & 0 & 1 & 0 & 1 & 0 & 0 & 0 \\
                  &   &   &   & E\\
            };

            \begin{pgfonlayer}{background}
                \draw[draw=blue!75,fill=blue!25,thick] (m-2-5.south west) rectangle (m-1-5.north east);
            \end{pgfonlayer}
        \end{tikzpicture}

        \begin{tikzpicture}
            \matrix (m) at (0,0) [matrix of nodes] {
                0 & 0 & 1 & 0 & 1 & 0 & 0 & 0 \\
                  &   &   &   &   & E\\
            };

            \begin{pgfonlayer}{background}
                \draw[draw=blue!75,fill=blue!25,thick] (m-2-6.south west) rectangle (m-1-6.north east);
            \end{pgfonlayer}
        \end{tikzpicture}

        \begin{tikzpicture}
            \matrix (m) at (0,0) [matrix of nodes] {
                0 & 0 & 1 & 0 & 1 & 1 & 0 & 0 \\
                  &   &   &   & A &  \\
            };

            \begin{pgfonlayer}{background}
                \draw[draw=blue!75,fill=blue!25,thick] (m-2-5.south west) rectangle (m-1-5.north east);
            \end{pgfonlayer}
        \end{tikzpicture}

        \begin{tikzpicture}
            \matrix (m) at (0,0) [matrix of nodes] {
                0 & 0 & 1 & 0 & 0 & 1 & 0 & 0 \\
                  &   &   & B &   &  \\
            };

            \begin{pgfonlayer}{background}
                \draw[draw=blue!75,fill=blue!25,thick] (m-2-4.south west) rectangle (m-1-4.north east);
            \end{pgfonlayer}
        \end{tikzpicture}

        \begin{tikzpicture}
            \matrix (m) at (0,0) [matrix of nodes] {
                0 & 0 & 1 & 0 & 0 & 1 & 0 & 0 \\
                  &   & C &   &   &  \\
            };

            \begin{pgfonlayer}{background}
                \draw[draw=blue!75,fill=blue!25,thick] (m-2-3.south west) rectangle (m-1-3.north east);
            \end{pgfonlayer}
        \end{tikzpicture}

        \begin{tikzpicture}
            \matrix (m) at (0,0) [matrix of nodes] {
                0 & 0 & 1 & 0 & 0 & 1 & 0 & 0 \\
                  & C &   &   &   &  \\
            };

            \begin{pgfonlayer}{background}
                \draw[draw=blue!75,fill=blue!25,thick] (m-2-2.south west) rectangle (m-1-2.north east);
            \end{pgfonlayer}
        \end{tikzpicture}

        \begin{tikzpicture}
            \matrix (m) at (0,0) [matrix of nodes] {
                0 & 1 & 1 & 0 & 0 & 1 & 0 & 0 \\
                  &   & D &   &   &  \\
            };

            \begin{pgfonlayer}{background}
                \draw[draw=blue!75,fill=blue!25,thick] (m-2-3.south west) rectangle (m-1-3.north east);
            \end{pgfonlayer}
        \end{tikzpicture}

        \begin{tikzpicture}
            \matrix (m) at (0,0) [matrix of nodes] {
                0 & 1 & 1 & 0 & 0 & 1 & 0 & 0 \\
                  &   &   & D &   &  \\
            };

            \begin{pgfonlayer}{background}
                \draw[draw=blue!75,fill=blue!25,thick] (m-2-4.south west) rectangle (m-1-4.north east);
            \end{pgfonlayer}
        \end{tikzpicture}

        \begin{tikzpicture}
            \matrix (m) at (0,0) [matrix of nodes] {
                0 & 1 & 1 & 0 & 0 & 1 & 0 & 0 \\
                  &   &   &   & E &  \\
            };

            \begin{pgfonlayer}{background}
                \draw[draw=blue!75,fill=blue!25,thick] (m-2-5.south west) rectangle (m-1-5.north east);
            \end{pgfonlayer}
        \end{tikzpicture}

        \begin{tikzpicture}
            \matrix (m) at (0,0) [matrix of nodes] {
                0 & 1 & 1 & 0 & 1 & 1 & 0 & 0 \\
                  &   &   & A &   &  \\
            };

            \begin{pgfonlayer}{background}
                \draw[draw=blue!75,fill=blue!25,thick] (m-2-4.south west) rectangle (m-1-4.north east);
            \end{pgfonlayer}
        \end{tikzpicture}

        \begin{tikzpicture}
            \matrix (m) at (0,0) [matrix of nodes] {
                0 & 1 & 1 & 0 & 1 & 1 & 0 & 0 \\
                  &   &   & F &   &  \\
            };

            \begin{pgfonlayer}{background}
                \draw[draw=blue!75,fill=blue!25,thick] (m-2-4.south west) rectangle (m-1-4.north east);
            \end{pgfonlayer}
        \end{tikzpicture}
    \end{center}
    
    The F state is special because it does not trigger any new instructions, so the machine halts once it reaches this state.
    Looking at the result of the individual steps, we can now see that the instructions at the beginning of the example program the Turing machine so that it copies sequences of 1s.
    If the tape had contained 11111 instead of 11, the final output would have contained two instances of 11111, assuming a sufficiently long tape.
\end{examplebox}

The example above shows that a Turing machine can compute copies.
As we will see in later chapters, copying is actually a very complex task that nonetheless plays an important role in natural languages.
Despite the complexity of copying, it can be understood in the very general terms of a Turing machine as simply a sequence of configuration changes: what does the tape look like, where are we on the tape, and what state is the machine in?

The generality of the Turing machine is what enables a broader understanding of computation that does not care about the actual hardware.
A Turing machine is not a concrete object, it's not a tiny box with some tape and a state dial that you can order from Amazon.
Instead, it is an abstract model of what it means to carry out a computation, and there are many different ways a Turing machine can be instantiated in the real world.
This is particularly noteworthy because Turing machines act as a kind of standard model for computation.
It is known that if there is no limit on how much tape is available, any problem that can be solved computationally can be solved by a Turing machine.
So all kinds of computation can be regarded as a specific program that runs on a Turing machine.
But this also means that any machine, system, or construct that provides the equivalent of a tape and a controllable read-write head is a computing device.

For example, a Turing machine can even take the form of a very selfish drinking game:
Gather 4 friends of yours and 6 shot glasses --- I am boldly assuming that you have enough of both.
Put the shot glasses in a line and fill the rightmost two with a beverage of your choice.
Then give each one of your friends a card with instructions they have to follow.
For the sake of exposition, let's assume that your friends are called Bill, Cathy, Damian, and Edith.
Their respective cards read as follows:
%
\begin{itemize}
    \item \textbf{Bill}\\
        If the shotglass in front of you is empty, get out of line and put Cathy in front of the glass to the left.
        Otherwise, leave the glass alone (sorry!), and move to the glass to the left.

    \item \textbf{Cathy}\\
        If the shotglass in front of you is empty, fill it up, get out of line, and put Damian in front of the glass to the right.
        Otherwise, leave the glass alone (sorry!),  and move to the glass to the left.

    \item \textbf{Damian}\\
        If the shotglass in front of you is empty, get out of line and put Edith in front of the glass to the right.
        Otherwise, move to the glass to the right.

    \item \textbf{Edith}\\
        If the shotglass in front of you is empty, fill it up, get out of line, and put me in front of the glass to the left.
        Otherwise, move to the glass to the right.
\end{itemize}
%
The instructions for yourself are slightly more fun.
If the glass in front of you is full, drink it all, get out of line, and put Bill in front of the glass to the left.
If the glass is empty, the game is over.

I suppose you can already tell what is going on here.
When you play this game, it will proceed exactly like the Turing machine from our copying example (it is a selfish drinking game because you are the only one who gets to drink, i.e.~rewrite a 1 as a 0).
Even though you and your friends are separate individuals and do not form any kind of unified object, the combination of all of you and the shotglasses constitute a Turing machine.

We can make all kinds of changes to this setup without losing the connection to Turing machines.
For example, we may use bowls instead of glasses, and fill them with M\&Ms instead of some beverage.
And maybe we do not actually put the bowls in a line but instead assume that they all differ in size like a set of baking bowls, so that ``left'' means ``the next smaller one'' and ``right'' means ``the next bigger one''.
Perhaps we could even replace your friends with a very well-trained dog.
Clearly a dog and a human are two very different things, but it changes nothing about the computation that is being carried out.
No matter how we set things up, the same input will always be transformed into the same output.

Silly as these example may be, the central point stands: changes to the physical make-up of the device that carries out the computations does not entail changes to the computation itself.
The notion of computation operates at a higher level of abstraction, and that is what gives it such a unifying power.
We can take computational concepts and apply them to computing devices that do not at all look like the computers we are familiar with: the human brain, the biological mechanisms of gene expression, even the universe itself.
Despite the differences in physical substrates, structural changes, and sheer computing power, they are all equally valid examples of computing devices and we can  discover interesting things about them by adopting this perspective.

Hopefully you can now appreciate why it is unfortunate that the term \emph{computational linguistics} does not clearly disambiguate between computation and computers.
\Note{At least it is better than the German term \emph{Computerlinguistik}, which can only have the second meaning when interpreted compositionally.}
While the latter emphasizes engineering concerns, the former strives for a more abstract perspective that applies to computers as well as humans.
Seeing how humans are the only known computing device with excellent command of natural language, we would do well to study language at a level that is compatible with those devices and learn from them as much as possible.
To clearly differentiate the two notions of computational linguistics, I will use \emph{natural language processing} (NLP) in this book to refer to those aspects of computational linguistics that are solely concerned with computers.
To summarize: NLP is about solving language-related tasks with computers, e.g.~speech synthesis, machine translation, or even the basic search function in your text editor;
computational linguistics is about studying language as an instance of computation.

% Parental advisory: a beverage with high alcohol content may cause premature failure of the experiment.
% The same computation will look very different when executed by a human brain, with neurons firing in a specific cascade that gives rise to a three-dimensional activation pattern.
% Or maybe we are just dealing with a few beads being moved around in an abacus.

This is all still very vague, and to some extent things will only get clearer once we move on from our high-level vantage point and start looking at concrete issues.
Still, it is always nice to have a rough idea of the road ahead and why it is road worth taking.
So let us next look at why it makes sense to study language from a computational perspective and how one might go about this.


\section{Computational linguistics: The what, how, and why}

\subsection{Practical arguments}

It seems fairly easy to make a case for the importance of studying computational aspects of language (putting aside for now what exactly we mean by that). 
After all, a world in which computers can successfully handle all kinds of language-related tasks is preferable to one where they cannot.
This would create a second industrial revolution that boldly pushes automation into language-heavy domains: customer service and speech-driven user interfaces, language and writing instruction, knowledge aggregation, and much more.
Admittedly there is also the risk of mass surveillance, mass unemployment, and the social upheavals that tend to follow both, but one could argue that those would just be short-term growing pains on the way to a more prosperous future.
If this is correct, then it is imperative that we do whatever we can to get computers to this level of aptitude.
And just like some understanding of physics had to be in place before engineers could bless mankind with the radio or the combustion engine, we cannot have successful NLP applications without a minimum understanding of language and the computational challenges it poses.
Computational linguistics thus is a prerequisite for NLP\@.

This argument, while intuitively pleasing, is too simplistic.
One of the most shocking aspects of the applied sciences and engineering is how little genuine understanding one needs to construct a useful tool.
To give but one example: relativity theory is not an integral part of calculating artillery ballistics.
In many areas of life the permitted margin of error is large enough that shortcuts, hacks, and brute force methods will get the job done just fine.
For practical purposes it is also perfectly fine to make stipulations that fly in the face of scientific consensus but improve the final results.
In the words of Noam Chomsky \citep[147]{Chomsky90}:
%
\begin{quote}
    Throughout history, those who built bridges or designed airplanes often had to make explicit assumptions that went beyond the understanding of the basic sciences.
\end{quote}

Similar things can be observed in NLP\@.
Many of its tools and techniques ignore well-established linguistic knowledge for the sake of simplicity and efficiency.
Nonetheless these tools do surprisingly well and often outperform competing models that draw from what linguists have learned about language.
The computational linguist Frederick Jelinek summarized this state of affairs very succinctly during his time at IBM in the 80s: ``Every time I fire a linguist, the performance of the speech recognizer goes up''.
A more recent example is a study carried out by Choi \& XXX at Stony Brook University, which evaluated the ability of various models to predict the success of novels.
% fixme: fix names
Naively, one would expect that a novel's success is largely dependent on its writing style, narrative structure, and subject matter.
Yet one of the best models completely ignored those aspects and focused exclusively on word frequencies.
At this point, then, there is still a large gap between ``language as a computational problem'' and ``computers solving natural language tasks''.

With the rapid rise of machine learning methods in recent years, the gap may even widen over the next few decades.
General purpose machine learning strategies frequently result in better models than those produced by language-specific techniques in NLP\@.
If this trend continues to its logical extreme, NLP may be completely absorbed into the field of machine learning.
In this case, any language-specific insights would be even harder to incorporate given the domain-agnostic nature of the machine learning techniques.



Things might change in the future as users demand more and more accurate tools for increasingly difficult tasks, but at this stage there is still a large gap between ``language as a computational problem'' and ``computers solving natural language tasks''.

A more refined version of the practical applicability argument points out that while a theoretically informed perspective may not be of any practical use for now, it can lay the foundation for entirely new fields, tools, and businesses in the future.
The poster child for this is of course number theory, which for the longest time was considered a purely theoretical and utterly useless subfield of mathematics but is now the central pillar on which all of modern cryptography rests.
Whether you are encrypting files on your computer or talking to a server through a safe channel, it all builds on number theory.
But we do not have to look at mathematics to see examples of theoretical research giving rise to important new developments, linguistics has several examples of its own.
Formal language theory --- which is an integral part of computer science and indispensable for the design of programming languages and file standards like XML, among other things --- grew out of Chomsky's attempts to develop formal models of natural language syntax.
Chomsky's writings about syntactic transformations also served as an inspiration for William Rounds' \citep{Rounds70} work on tree transducers, which are now used in compiler design and machine translation.
More recently, Aravind Joshi's Tree Adjoining Grammar (TAG) has even been used to model messenger RNA \citep{Uemura.etal99, Matsui.etal05}.
Overall, then, there is ample evidence that theoretical inquiry does benefit practical applications eventually and consequently we may hope that studying the computational aspects of language will benefit not only NLP, but also future areas that we cannot envision at this point.

\subsection{Scientific Arguments}

Utilitarian arguments may convince politicians, taxpayers, and engineers, but they hold no sway in the court of science.
A linguist would shrug at the arguments above on a good day, and publicly denounce us as nimrods on a bad one.
Fortunately, the scientific merit of computational linguistics is a lot more clear-cut: language is intrinsically a computational problem, so it should be studied as such.

\Note{Semantics is noteworthy for being the one subfield of linguistics that is still very strongly aligned with the externalist view of language and does not think of its formalism as the high-level description of a specific part of human cognition.
For a mentalist approach to natural language meaning see \citet{Pietroski14}.
}
Probably the most important development in 20\tsp{th} century linguistics is the \emph{cognitive turn}, the shift from viewing language as an external system of rules and words to its reinterpretation as one of humans' many cognitive abilities.
Language is not an abstract platonic object, it is done by humans, it is an algorithm that runs on the human brain (the \emph{wetware}).
That immediately raises the question how language is computed.

Linguists actually split this up into two subquestions:
%
\begin{description}
    \item[Competence] What is the specification of the computations?
    \item[Performance] How is this specification implemented and used?
\end{description}
%
Competence questions are concerned with the rules of grammar and how they are encoded.
Somewhat sloppily, one could describe competence as ``language modulo resource restrictions like limited working memory and attention span''.
Of course nothing of this sort can be observed in nature --- competence is an artificial abstraction of performance, which is about how the specification behaves when it is run on an actual machine, i.e.\ the human brain.
The distinction between competence and performance also exists in computer science to some extent.
For example, complexity theory studies the difficulty of problems of unbounded size, even though in practice problems are usually bounded, e.g.\ because a computer can only store so much information.
Nonetheless complexity theory has produced results that are also relevant in practice, and competence questions in linguistics have similarly shed some light on performance.

Since competence cannot be directly observed, research into how language is computed usually operates in the realm of performance.
Linguists approach this questions through the lens of neuroscientists and psychologists: how does the wetware behave when carrying out specific linguistic tasks, and can we design a procedure that mimics humans' behavior?
For example, native speakers of English usually have no problem understanding English sentences, it is an incredibly fast and effortless process.
But it does exhibit surprising quirks.
When asked whether the sentence \eqref{ex:Formal_GardenPath} is grammatical, they usually say yes.
%
\begin{exe}
        \ex The player tossed a frisbee smiled.\label{ex:Formal_GardenPath}
\end{exe}
%
However, the sentence is actually grammatical, it has the same structure as the minimally different \eqref{ex:Formal_Non-GardenPath}. 
%
\begin{exe}
    \ex The player thrown a frisbee smiled.\label{ex:Formal_Non-GardenPath}
\end{exe}
%
For some reason the algorithm native speakers of English use has no problem with \eqref{ex:Formal_Non-GardenPath} but is completely thrown off by the exchange of a single word that serves exactly the same grammatical function.
This is almost like a computer that can compute $1 + 2$ but not $2 + 1$.
There is no obvious reason for this behavior, and it doesn't exactly look like good engineering (so much for intelligent design).
Linguists have come up with various elaborate explanations of such phenomena over the years --- some more successful than others ---
but they all share a variety of properties that necessarily limit their scope and thus the questions they can address.

\subsubsection{Neuroscience}

It is certainly interesting to see how the human brain works on the hardware level, but that does not tell us anything about the actual computations --- just like studying a computer's hardware tells us little about what it is computing.
If we hear the hard drive spin up we can make the reasonable assumption that some file is being accessed, but that is about all we can deduce with our own senses.
Inquisitive minds that we are, we may then decide to connect a set of thermal diodes to various points of the hardware in order to detect whether that piece of hardware starts heating up, suggesting an increase of computational activity there.
But that is still very inconclusive, for various reasons.

First of all, a higher load on the graphics chip might indicate that some kind of 3D graphics is being rendered, e.g.\ for a video game, but actually a lot of non-graphical tasks are outsourced from the processor to the graphics chip nowadays.
\Note{Graphics chips are also called Graphics Processing Units (GPUs).
    The usage of GPUs for non-graphical tasks is known as GPGPU: General Purpose Computing on GPUs.}
And even if it were possible to tie each piece of hardware to a specific task,
we still could not determine whether the computer is, say, sorting a list or searching through it, two very different tasks.
More fine-grained distinctions are completely unthinkable, such as whether one of the two search algorithms below is being used, and if so, which one.
%
\begin{center}
    \pythonfile[firstline=5,firstnumber=1]{./code/list_search/linear_search.py}
\end{center}
\begin{center}
    \begin{pythoncode}
        >>> test_list = ['a', 'c', 'd', 'e', 'b', 'f']
        >>> linear_search(test_list, 'e')
        3
        >>> linear_search(test_list, 'g')
        False
    \end{pythoncode}
\end{center}

\begin{center}
    \pythonfile[firstline=8,lastline=51,firstnumber=1]{./code/list_search/binary_search_loop.py}
\end{center}
\begin{center}
    \begin{pythoncode}
        >>> test_list = ['a', 'c', 'd', 'e', 'b', 'f']
        >>> binary_search(sorted(test_list), 'e')
        4
        >>> binary_search(sorted(test_list), 'g')
        False
        >>> binary_search(test_list, 'c')
        1
        >>> binary_search(test_list, 'e')
        False
    \end{pythoncode}
\end{center}
%
The binary search algorithm is a lot more sophisticated than linear search, and it is also much faster on average.
It does have the disadvantage that it only works for sorted lists, so unsorted lists need to be sorted first, and in cases where no sorting order can be defined, the algorithm does not work at all.
So if we had two computers with exactly the same hardware, one using linear search and the other binary search, the latter would show vastly superior performance in general but would fail miserably in specific cases.
The behavior of this computer might even look similar to humans' puzzling problems with certain grammatical sentences.
Yet the behavior cannot be explained in terms of hardware, because we know for a fact that the two machines are exactly the same.

In the case of computers, we have the advantage that we already know how their architecture works because we designed it.
So if that still isn't enough to deduce from the hardware what kind of computations a computer is carrying out, it seems rather unlikely that a similar process could derive from wetware the functioning of the human brain.
Granted, we can uncover limiting factors and some basic facts, just like a close analysis of a processor can reveal a maximum limit on its working memory (Level 1, 2, and 3 caches) and that all information is encoded in a categorical fashion via series of on and off states --- the proverbial 0s and 1s.
But all the probing, all the high-tech machinery tells us very little about the actual computations.
That does not mean it is a fruitless endeavor, and eventually it will be possible to connect these two levels via some linking theory, but the crucial aspect is that we do need both levels, neuroscience alone cannot give us the full picture of language as a cognitive ability.


\subsubsection{Psychology}

Psychologists aren't interested in the physical instantiation of cognition, they are perfectly happy to treat the brain as a black box that produces a certain output given a certain input.
Their goal is to develop models that replicate these input-output mappings.
But more often than not these models are highly specific in what cognitive parameters they presuppose.

A very common assumption is that humans use \emph{content addressable memory} (CAM).
In contrast to computer's \emph{random accessible memory} (RAM), information stored in CAM is not retrieved via an address that specifies its precise location in memory.
Instead, the individual pieces of the information themselves act as a way of specifying the path to its location in memory.
Imagine traversing a network where \emph{doctor} takes you in one direction, and then \emph{crab} in another so that you end up at a node that stores all the information about Doctor John A.\ Zoidberg from \emph{Futurama}.
Memory is also considered to be very limited in size and possibly stratified according to certain types of content.
These properties are then coupled with certain models of memory activation and retrieval to explain specific aspects of human cognition, e.g.\ priming effects.

The downside of this approach is that it relies on assumptions that are hard if not even impossible to prove conclusively, that there's often many alternative solutions with no obvious way of choosing between them, and that complex accounts --- by virtue of being complex --- make it very hard to assess how much work is done by each individual part.
The lack of conclusive proof is not too much of an issue, uncertainty is a given in almost every scientific endeavor.
One cannot help but wonder, though, if the degree of uncertainty could be lowered.
If there is a higher-level explanation that does not rely on quite as specific assumptions about working memory, that should be preferable.
The same goes for the problem of too many solutions: if you can cook up multiple models that use slightly different memory architectures but yield the same result when each one of them is also coupled with slightly different amount of memory, that indicates that the essential property may be more abstract.
The proposed psychological model would then be but one of many different ways of enforcing this property.
And since there are usually many alternative solutions, it is very hard to show that certain parts of the machinery are indispensable to get the desired result.

\subsubsection{Interim Summary and a Promise}

All the limitations of neuroscience and psychology pointed out so far can be reduced to a lack of abstraction.
Both fields operate at levels that specify a lot of information --- like memory addressing and neural connections --- whose relevance to language isn't apparent; in particular if one cares mostly about competence questions, as most theoretical linguists do.
The great promise of computational linguistics, the one advantage that sets it apart from neuroscience and psychology, is that it can completely abstract away from all extraneous detail and performance aspects.
The methods used by computational linguists allow us to connect language and computation at the competence level. 
We can conclusively answer questions such as
%
\begin{itemize}
    \item What is the weakest memory architecture that is sufficiently powerful to support a specific model of competence?
    \item What is the weakest competence model that is sufficiently powerful for a given empirical domain, e.g.\ local processes in phonology.
    \item Do alternative competence models describe the same class of computations?
    \item Is there an alternative representation of a given model that lowers memory requirements?
    \item How can we carve up complex models into simple subparts? 
    \item What kind of computational universals hold of language?
\end{itemize}


\section{Marr's Three Levels and the Virtue of Abstractness}

The observations made so far are far from new, they were already encompassed in \emph{Marr's three levels of analysis} \citep{MarrPoggio76}.
%fixme: Marr reference
Marr proposes that any aspect of cognition can be described on three levels of increasing abstraction:
%
\begin{description}
    \item[physical] the wetware or hardware instantiation; e.g.\ neural circuitry for vision or the machine code running on your computer
    \item[algorithmic] what kind of computational steps does the system carry out and in which order, what are its data structures and how are they manipulated; basically the level of programming
    \item[computational] what problem space does the system operate on, how are the solutions specified
\end{description}
%
\begin{examplebox}[Set Intersection on Three Levels]
    Suppose you have two sets of objects, $A$ and $B$, and you want to write a computer program that tells you which objects belong to both sets.
    \begin{itemize}
        \item On a computational level, that program is easily specified: it takes two sets as input and returns their intersection ($A \cap B$).
        \item On the algorithmic level, things get trickier.
        For instance, what kind of data structure do you want to use for the input (sets, lists?), and just how does one actually construct an object that's the intersection of two sets?
    \item On the physical level, finally, things are so complicated that it's basically impossible to tell what exactly is being computed by the machine.
        Voltages increase or decrease in various transistors spread over the CPU, memory and mainboard, and that's about all you can make out.
        Unless you already have a good idea of the higher levels and the computational process being carried out, it's pretty much hopeless to reverse engineer the program from the electrical signals.
    \end{itemize}
\end{examplebox}
%
Marr's levels of analysis highlight that one and the same object can be described in very different ways, and all three levels are worth studying.
A computational specification can be implemented in various distinct ways on the algorithmic level, and algorithms can be realized in a myriad of physical ways --- for instance, your laptop and your tablet use very different processor architectures (x86 and ARM, respectively), but a Python program will run just fine on either platform despite the differences in electric signals.
And of course this hierarchy is continuous: Assembly code is closer to the physical level than C, which in turn is closer to it than Python.
However, the more you are interested in stating succinct generalizations, the more you'll be drawn towards abstractness and hence the computational level at the top of the continuum.
And this is exactly the level computational linguistics is aiming for.

\section{Closing Remark: The Need for Formalization}

The problem with abstraction is that one can no longer reason on a purely intuitive level.
Since the objects are characterized by a few basic properties, it is important that these properties are described as precisely as possible.
A minor misunderstanding may be enough to lead to completely contradictory conclusions.
In the worst case, we may end up with an \emph{inconsistent} theory, meaning that there is at least one property that is both true and false at the same time.
This may be perfectly fine in a post-modern analysis of transphobic slurs in humoristic epitaphs, but it has no place in a scientific theory.
So abstraction necessarily requires a certain degree of rigor.

This shouldn't come as a big shock to you.
Computer science can be very rigorous, in particular its theoretical subfields like complexity theory and formal language theory.
The same is also true of linguistics:
Generative syntax and phonology are abstract and involve a lot of technical machinery that seems arcane and intimidating to outsiders.
The technical machinery is indispensable for each field's areas of scientific inquiry, and you all got the hang of it eventually after a few initial struggles.

The same is true of the machinery we will use in this course.
It is more technical than linguistics, but only because we cannot make do with less.
It does involve some math, but nothing that one couldn't pick up in a week.
It is harder to read at the beginning, but I will try to keep notation to a minimum.
You will get stuck sometimes, but that just means you have to think about the problem a couple more times until you get it.
\Note{%
    If you don't know it yet, I highly recommend that you read Martin Schwarz's excellent essay \href{http://jcs.biologists.org/content/121/11/1771.full}{The importance of stupidity in scientific research}.
And if you do know it, I highly recommend that you read it again.
}
Nothing we do in here is really difficult, but it takes patience and dedication.
If some of the material we cover seems overwhelming to you, remember: the most important trait of a good researcher is to enjoy feeling stupid.

\section*{Central concepts}

\begin{itemize}
    \item Computation $\neq$ computers
    \item Turing machines
    \item Linear VS binary search
    \item Marr's levels of description
    \item Virtue of abstraction
\end{itemize}

\section*{References and recommended readings}

Manning on machine learning
SIGMORPHON competition with neural networks
Heinz on SL versus LSTM
Sean Fulop quote


\part{Phonology}
\include{./tex/listphonology}
\include{./tex/strictlylocal}
\include{./tex/sl-mathematics}
\include{./tex/sl-alternatives}
\include{./tex/learnabilitysl}
\include{./tex/probabilisticlocal}
\include{./tex/strictlypiecewise}
\include{./tex/regular}
\include{./tex/spe}

\part{Morphology and Syntax}
% \include{./tex/morphology}
% \include{./tex/trees}
% \include{./tex/generativecapacity}
% \include{./tex/syntacticcomplexity}
% \include{./tex/parsing}
% \include{./tex/fosyntax}
% \include{./tex/derivations}
% \include{./tex/epilogue}

\printbibliography[heading=bibintoc]
% \bibbysegment[heading=subbibliography]
\end{document}
